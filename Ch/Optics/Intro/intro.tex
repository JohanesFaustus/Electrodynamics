\documentclass[../../../main.tex]{subfiles}
\begin{document}
\subsection*{Fermat's Principle}
\subsubsection*{Optical path.} Corresponds to the distance in vacuum equivalent to the distance traversed in the medium of index $n$. An  optical path form point $S$ to point $P$ is defined as 
\begin{equation*}
    OPL=\int_{S}^{P}n(s)\;ds
\end{equation*}

\subsubsection*{Fermat's principle.} State that light will travel the route such that $OPL$ in minimum 

\subsection*{Law of Reflection}
State that the angle-of-incidence equals the angle-of-reflection
\begin{equation*}
    \theta_i=\theta_r
\end{equation*}
\subsubsection*{Derivation.} 404 not found.

\subsection*{Snell’s Law}
Also called law of refraction
\begin{equation*}
    n_i\sin\theta_i = n_r\sin\theta_r
\end{equation*}

\subsubsection*{Derivation.} 404 not found.

\subsubsection*{Total internal reflection.} In the case of $n_i>n_r$, when the incidence angel $\theta_i$ is equal or greater than the critical angle $\theta_c$, total internal reflection occur. Snell's law state 
\begin{equation*}
    \sin\theta_i =\frac{n_r}{n_i}\sin\theta_r
\end{equation*}
The critical angel occur when the reflected angle is perpendicular to normal, then said incidence angle is defined as critical angle
\begin{equation*}
    \sin\theta_i =\frac{n_r}{n_i}\implies \theta_c=\arcsin \frac{n_r}{n_i}
\end{equation*}

\begin{figure}[b]
    \centering
    \normfigXL{../../../Rss/Optics/Intro/TotalInternalDeath.png}
    \caption*{Figure: total internal reflection}
\end{figure}
\end{document}