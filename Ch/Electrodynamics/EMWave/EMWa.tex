\documentclass[../../../main.tex]{subfiles}
\begin{document}
\subsection*{Wave Equation}
The wave equation for electromagnetic waves are 
\begin{equation*}
    \nabla^2\mathbf{E}=\mu_0\epsilon_0\frac{\partial^2\mathbf{E}}{\partial t^2},\quad\text{and}\quad \nabla^2\mathbf{B}=\mu_0\epsilon_0\frac{\partial^2\mathbf{B}}{\partial t^2},
\end{equation*}
Comparing to the classical wave equation, we have the speed at which the waves travel
\begin{equation*}
    v=\frac{1}{\sqrt{\mu_0\epsilon_0}}
\end{equation*}

\subsubsection*{Derivation.}
The Maxwell equations in empty space transform
\begin{gather*}
    \nabla\cdot\mathbf{E}=0\\
    \nabla\cdot\mathbf{B}=0\\
    \nabla\times\mathbf{E}=-\frac{\partial \mathbf{E}}{\partial t}\\
    \nabla\times\mathbf{B}=\mu_0\epsilon_0\frac{\partial \mathbf{E}}{\partial t}
\end{gather*}
Applying curl into the curl of $\mathbf{E}$
\begin{equation*}
    \nabla\times\nabla\times\mathbf{E}=\nabla(\nabla\cdot\mathbf{E})-\nabla^2\mathbf{E}=-\nabla^2\mathbf{E}
\end{equation*}
We then equate it with 
\begin{equation*}
    \nabla\times\nabla\times\mathbf{E}=\nabla\times\left(-\frac{\partial \mathbf{B}}{\partial t}\right)=-\frac{\partial}{\partial t}\nabla\times\mathbf{B}=-\mu_0\epsilon_0\frac{\partial^2\mathbf{B}}{\partial t^2}
\end{equation*}
to obtain the wave equation for electric wave 
\begin{equation*}
    \nabla^2\mathbf{E}=\mu_0\epsilon_0\frac{\partial^2\mathbf{E}}{\partial t^2}    
\end{equation*}
For the case of magnetic wave, we apply the curl into the curl of $\mathbf{B}$
\begin{equation*}
    \nabla\times\nabla\times\mathbf{B}=\nabla(\nabla\cdot\mathbf{B})-\nabla^2\mathbf{B}=-\nabla^2\mathbf{B}
\end{equation*}
and equate it with
\begin{equation*}
    \nabla\times\nabla\times\mathbf{B}=\nabla\times\left(\mu_0\epsilon_0\frac{\partial \mathbf{E}}{\partial t}\right)=\mu_0\epsilon_0\frac{\partial}{\partial t}\nabla\times\mathbf{E}=-\mu_0\epsilon_0\frac{\partial^2\mathbf{B}}{\partial t^2}
\end{equation*}
to obtain
\begin{equation*}
    \nabla^2\mathbf{B}=\mu_0\epsilon_0\frac{\partial^2\mathbf{B}}{\partial t^2}    
\end{equation*}

\subsection*{Monochromatic Plane Waves}
The field of monochromatic plane waves are expressed as 
\begin{gather*}
    \mathbf{E}(z,t)=\mathbf{E}_0e^{i(kz-\omega t)}\\
    \mathbf{B}(z,t)=\mathbf{B}_0e^{i(kz-\omega t)}
\end{gather*}
This wave will propagate on the $z$ axis, but we can generalize it using wave vector $\mathbf{k}$ which points to direction of propagation and has the magnitude of wave number $k$
\begin{equation*}
    \mathbf{E}(\mathbf{r},t)=E_0e^{i(\mathbf{k}\cdot\mathbf{r}-\omega t)}\;\mathbf{\hat{n}},\quad\text{and}\quad
    \mathbf{B}(\mathbf{r},t)=B_0e^{i(\mathbf{k}\cdot\mathbf{r}-\omega t)}\;\mathbf{\hat{n}}
\end{equation*}
where $\mathbf{\hat{n}}$ is the polarization vector and that $\mathbf{\hat{n}}\cdot\mathbf{\hat{k}}=0$ since electromagnetic waves are transversal.

Maxwell equation also acts as constrain that yield the relation 
\begin{equation*}
    \mathbf{B}=\frac{1}{c}\;\mathbf{\hat{z}}\times\mathbf{E}_0
\end{equation*}
or the general form 
\begin{equation*}
    \mathbf{B}(\mathbf{r},t)=\frac{1}{c}\;\mathbf{\hat{k}}\times\mathbf{E}(\mathbf{r},t)
\end{equation*}

\subsubsection*{Derivation.}
Evaluate the curl of electric field for this case 
\begin{align*}
    \nabla\times\mathbf{E}&=
    \begin{vmatrix}
         \mathbf{\hat{x}}& \mathbf{\hat{y}}& \mathbf{\hat{z}}\\
        \dfrac{\partial }{\partial x}&\dfrac{\partial}{\partial y}&\dfrac{\partial }{\partial z}\\
        E_x&E_y&0
    \end{vmatrix}=-\frac{\partial E_y}{\partial z}\;\mathbf{\hat{x}}+\frac{\partial E_x}{\partial z}\;\mathbf{\hat{y}}\\
    \nabla\times\mathbf{E}&=-E_{0y}ike^{i(kz-\omega t)}\mathbf{\hat{x}}-E_{0x}ike^{i(kz-\omega t)}\mathbf{\hat{y}}
\end{align*}
and the changes in magnetic field
\begin{equation*}
    -\frac{\partial \mathbf{B}}{\partial t}=-B_{0x}i\omega e^{i(kz-\omega t)}\mathbf{\hat{x}}-B_{0y}i\omega e^{i(kz-\omega t)}\mathbf{\hat{y}}
\end{equation*}
According to Maxwell equation, we can equate both to obtain 
\begin{equation*}
    -E_{0y}k=B_{0x}\omega,\quad\text{and}\quad E_{0x}k=B_{0y}\omega
\end{equation*}
or 
\begin{equation*}
    \mathbf{B}_0=\frac{k}{\omega}\;\mathbf{\hat{z}}\times\mathbf{E}_0
\end{equation*}
since
\begin{align*}
    B_{0x}\mathbf{\hat{x}}+  B_{0y}\mathbf{\hat{y}}&= \frac{1}{c}\;\mathbf{\hat{z}}\times(E_{0x}\mathbf{\hat{x}}+  E_{0y}\mathbf{\hat{y}})\\
    B_{0x}\mathbf{\hat{x}}+  B_{0y}\mathbf{\hat{y}}&= \frac{1}{c}(E_{0x}\mathbf{\hat{y}}-  E_{0y}\mathbf{\hat{x}})
\end{align*}
\end{document}